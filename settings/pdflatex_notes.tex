% Файл: settings/pdflatex_notes.tex
% Основные настройки для документов LaTeX

% ========================================
% Пакеты сгруппированы по функционалу
% ========================================

% Основные настройки документа
\usepackage{geometry} % Настройки полей и страницы
\usepackage{array}
\usepackage{indentfirst} % Красная строка
\usepackage{setspace} % Межстрочный интервал
\usepackage{ulem} % Подчеркивания

% Графика и оформление
\usepackage[most]{tcolorbox} % Стильные боксы
\usepackage{graphicx, wrapfig, caption, placeins} % Работа с изображениями
\usepackage{float} % Позиционирование рисунков
\usepackage{subcaption} % Подписи к подрисункам
\usepackage{booktabs} % Профессиональные таблицы
\usepackage{longtable} % Длинные таблицы

% Математика
\usepackage{amsmath, amssymb, amsfonts, mathtools, commath, cancel}
\usepackage{bm} % Жирные математические символы
\usepackage{nicefrac} % Красивые дроби

% Заголовки и колонтитулы
\usepackage{titlesec} % Стили заголовков
\usepackage{fancyhdr} % Колонтитулы
\usepackage{abraces} % Фигурные скобки

% Гиперссылки (должен быть загружен последним!)
\usepackage{hyperref}
\usepackage{xurl} % Для переноса длинных URL
\usepackage{tikz}
\usetikzlibrary{shapes, arrows.meta, positioning}
\usepackage[style=gost-numeric, backend=biber, refsection=section]{biblatex}

\usepackage[utf8x]{inputenc}
\usepackage{cmap}
\usepackage[english,russian]{babel}

% Основной размер шрифта и межстрочный интервал
\renewcommand{\normalsize}{\fontsize{12pt}{13pt}\selectfont} % 12pt - размер, 14pt - межстрочный интервал

% ========================================
% Основные настройки
% ========================================

% Шрифты
\renewcommand{\rmdefault}{cmr} % Основной шрифт (Computer Modern)
\renewcommand{\sfdefault}{cmss} % Без засечек
\renewcommand{\ttdefault}{cmtt} % Моноширинный

% Абзацные отступы и интервалы
\setlength{\parindent}{1.5em} % Абзацный отступ
\setlength{\parskip}{0.5em} % Интервал между абзацами

% Настройка geometry
\geometry{
	a5paper,
	left=10mm,
	right=10mm,
	top=5mm,
	bottom=10mm,
	headheight=10mm,
	headsep=4mm,
	footskip=10mm,
	includeheadfoot
}

% Стили заголовков
\titleformat{\section}[block]{\bfseries\centering\newpage\huge}{\thesection.}{0.5em}{}[]
\titleformat{\subsection}[block]{\bfseries\centering}{\thesubsection.}{0.5em}{}[]
\titleformat{\subsubsection}[block]{\bfseries\centering}{}{0.5em}{}[]

% Настройка колонтитулов
\pagestyle{fancy}
\renewcommand{\headrulewidth}{1pt}
\fancyhead{}
\fancyhead[L]{\nouppercase{\leftmark}}
\fancyfoot[C]{\thepage}
\fancyfoot[R]{Скороходов С.А.}
\renewcommand{\sectionmark}[1]{}
\renewcommand{\subsectionmark}[1]{\markboth{#1}{}}

% Настройка математики
\numberwithin{equation}{subsection}
\mathtoolsset{showonlyrefs=false}

% Настройка изображений
\graphicspath{{./img/}}
\renewcommand{\figurename}{Рисунок }
\captionsetup{labelsep=space}

% Гиперссылки
\hypersetup{
	colorlinks=true,
	linkcolor=blue,
	filecolor=magenta,
	urlcolor=cyan,
	pdftitle={Заметки},
	pdfpagemode=FullScreen,
}

% ========================================
% Пользовательские команды
% ========================================

% Блоки определений
\newtcolorbox{tbox}[2][]{
	colback=white!98!black,
	colframe=white!80!black,
	fonttitle=\bfseries,
	coltitle=black,
	title={#2},
	breakable,
	standard,
	before={\addcontentsline{toc}{subsubsection}{\protect\numberline{}#2}}
}

% Теоремы и определения
\usepackage{amsthm}

\theoremstyle{definition}
\newtheorem{definition}{Определение}[section]

\theoremstyle{plain}
\newtheorem{theorem}{\bfseries Теорема}
\newtheorem{corollary}[theorem]{Следствие}
\newtheorem*{proof*}{Доказательство}

\theoremstyle{remark}
\newtheorem{remark}{Замечание}[section]
\newtheorem{example}{Пример}[section]

% Удобные математические операторы
\DeclareMathOperator{\grad}{grad}
\DeclareMathOperator{\divergence}{div}
\DeclareMathOperator{\rot}{rot}