\subsection{\textbf{Теорема о движении центра масс.}}

Центр масс системы материальных точек называется геометрическая точка такая, что $\vec{r}$ которой равен:
\[\begin{aligned}
	\vec{r}_c = \frac{\sum_{i=1}^{n} m_i \vec{r}_i}{M} &&&& M = \sum_{i=1}^{n} m_i
\end{aligned}\]

Рассчитаем скорость центра масс:
\begin{multline*}
	\vec{v}_c = \dd{\vec{r}_c}{t} = \frac{1}{M} \dd{}{t} \Big(\sum_{i=1}^{n} m_i \vec{r}_{i}\Big) = \frac{1}{M} \sum_{i=1}^{n} m_i \vec{v}_i = \frac{\vec{p}}{M} \Rightarrow \boxed{\vec{p} = M \vec{v}_c}
\end{multline*}

Центру масс можно приписать импульс всей системы:
\begin{align*}
	\dd{\vec{p}}{t} = \vec{F}_{\text{внеш}} && \vec{p} = M\vec{v}_c \quad\to\quad M\vec{a}_c = \vec{F}
\end{align*}

\begin{tbox}{Замечания}
	\begin{enumerate}
		\item Расположение центра масс не зависит от выбора точки:
		\[\vec{r}_i = \vec{r}_i^{\,\prime} + \vec{\rho}\]
		\[\vec{r}_i = \frac{1}{M} \sum_{i=1}^{n} m_i \vec{r}_i = \frac{1}{M} \sum_{i=1}^n m_i \vec{r}_i^{\,\prime} + \frac{1}{M} \sum_{i=1}^{n} m_i \vec{\rho} = \vec{r}_i^{\,\prime} + \vec{\rho}\]
		\item Внутренние силы не влияют на движение центра масс. Они воздействуют на него, но опосредованно;
		\item Теорема о движении центра масс предоставляет общее представление о системе. Для более детального анализа систему переводят в систему отсчета, связанную с центром масс, и исследуют ее более подробно.
	\end{enumerate}
\end{tbox}