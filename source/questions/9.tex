\subsection{\textbf{Уравнение вращательного движения твердого тела вокруг неподвижной оси. Момент инерции, примеры его вычисления.}}

\textbf{Твердое тело} -- это недеформированное тело или система материальных точек расстояние между которыми не изменяется в процессе движения.

Механика твердого тела является частным случаем механики системы материальных точек.

\textbf{Признаки:}
\begin{itemize}
	\item Недеформированный;
	\item Сплошное (Масса размазана непрерывно);
	\item Переходим от материальной точки к элементу массы ($\Sigma \to \int$).
\end{itemize}

\begin{tbox}{Условие равновесия твердого тела}
	\begin{align*}
		\begin{cases}
			m \dd{\vec{v}_0}{t} = \vec{F} \\
			\dd{\vec{N}}{t} = \vec{M}
		\end{cases} \Rightarrow
		\begin{matrix}
			\vec{F} = 0\\ \vec{M} = 0
		\end{matrix} + \begin{matrix}
		\vec{v}_c(0) = 0\\
		\vec{N}(0) = 0
		\end{matrix}
	\end{align*}
\end{tbox}
