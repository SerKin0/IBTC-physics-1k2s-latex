\subsection{Потенциальная энергия с.м.т. \textbf{Теорема об изменении механической энергии с.м.т. Условия сохранения механической энергии.}}

\subsubsection*{Теорема об изменении механической энергии с.м.т.}
Запишем теорему для \textit{материальной точки}:
\[W_{\text{п}}(\vec{r}) = A_{\vec{r} \to \vec{r}_0}^{\text{конс}} = \int_{r_0}^{r} F_r \, dr,\]
где $\vec{r}_0$ -- точка, в которой потенциальная энергия принимается за $0$.

\textbf{Консервативная сила} -- это сила, которая зависит только от положения (координат) частицы и ее работа не зависит от формы траектории частицы, а зависит только от начального или конечного положения.


\subsubsection*{Физический смысл потенциальной энергии}
Потенциальная энергия характеризует запас работы или возможную работу, которую совершает точка при переходе из одной точки в другую.



Распишем изменение кинетической энергии:
\[W_{\text{кин}} = A_{12}^{\text{всех сил}} = A_{12}^{\text{конс}} + A_{12}^{\text{неконс}} = -\Delta W_{\text{пот}} + A_{12}^{\text{неконс}}\]