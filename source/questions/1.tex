\subsection{\textbf{Теорема об изменении импульса с.м.т.} Условия сохранения импульса.}

Запишем теорему об изменении импульса для \textit{материальной точки}:
\[\begin{aligned}
	\vec{p} = m\vec{v} &&&& \dd{\vec{p}}{t} = \sum_{i=1}^{n} \vec{F}_i
\end{aligned}\]

Обобщим для системы. Импульсом системы материальных точек называется сумма импульсов всех точек входящих в систему:
\[\vec{p} = \sum_{i=1}^{n} \vec{p}_i\]

\textbf{Физический смысл:} Характерное поступательное движение системы как единого целого.
\begin{align*}
	\dd{\vec{p}_1}{t} &= \vec{f}_{21} + \vec{f}_{31} + \dots + \vec{f}_{i1} + \dots + \vec{f}_{n1} + \vec{F}_1; \\
	\dd{\vec{p}_i}{t} &= \vec{f}_{1i} + \dots + \vec{f}_{(i-1) \, 1} +\vec{f}_{(i+1) \, 1} + \dots + \vec{f}_{ni} + \vec{F}_i; \\
	\dd{\vec{p}_n}{t} &= \vec{f}_{1n} + \dots + \vec{f}_{in} + \dots + \vec{f}_{(n-1)n} + \vec{F}_n;
\end{align*}

Сложим попарно внутренние силы и сократим: \(\vec{f}_{ik} + \vec{f}_{ki} = 0\).
\[\sum_{i=1}^{n} \dd{\vec{p}_i}{t} = \sum_{i=1}^{n} \vec{F}_i\]

