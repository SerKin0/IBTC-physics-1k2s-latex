\subsection{\textbf{Теорема об изменении импульса с.м.т.} Условия сохранения импульса.}

Запишем теорему об изменении импульса для \textit{материальной точки}:
\[\begin{aligned}
	\vec{p} = m\vec{v} &&&& \dd{\vec{p}}{t} = \sum_{i=1}^{n} \vec{F}_i
\end{aligned}\]

Обобщим для системы. Импульсом системы материальных точек называется сумма импульсов всех точек входящих в систему:
\[\vec{p} = \sum_{i=1}^{n} \vec{p}_i\]

\textbf{Физический смысл:} Характерное поступательное движение системы как единого целого.
\begin{align*}
	\dd{\vec{p}_1}{t} &= \vec{f}_{21} + \vec{f}_{31} + \dots + \vec{f}_{i1} + \dots + \vec{f}_{n1} + \vec{F}_1; \\
	\dd{\vec{p}_i}{t} &= \vec{f}_{1i} + \dots + \vec{f}_{(i-1) \, 1} +\vec{f}_{(i+1) \, 1} + \dots + \vec{f}_{ni} + \vec{F}_i; \\
	\dd{\vec{p}_n}{t} &= \vec{f}_{1n} + \dots + \vec{f}_{in} + \dots + \vec{f}_{(n-1)n} + \vec{F}_n;
\end{align*}

Сложим попарно внутренние силы и сократим: \(\vec{f}_{ik} + \vec{f}_{ki} = 0\).
\[\sum_{i=1}^{n} \dd{\vec{p}_i}{t} = \sum_{i=1}^{n} \vec{F}_i\]

Где $\vec{F}_{\text{внеш}} = \sum_{i=1}^{n} \vec{F}_i$ -- суммарная внешняя сила. Из чего следует:
\[\sum_{i=1}^{n} \dd{\vec{p}_i}{t} = \dd{}{t} \sum_{i=1}^{n} \vec{p}_i = \dd{\vec{p}}{t} \quad\Rightarrow\quad \boxed{\dd{\vec{p}}{t} = \vec{F}_\text{внеш}}\]

Запишем в интегральной форме:
\[\boxed{\vec{p}(t) - \vec{p}(t_0) = \int_{t_0}^{t} \vec{F}_{\text{внеш}} \, dt}\]

\begin{tbox}{Замечания}
	\begin{enumerate}
		\item Теорема справедлива и в НСО\footnote{Неинерциальная система отсчета (НСО) — система отсчета, движущаяся с ускорением относительно инерциальной. Простейшими НСО являются системы, движущиеся ускоренно прямолинейно, и вращающиеся системы.} (нужно учесть силы инерции);
		\item Описывает поступательное движение в целом и не включает внутреннюю силу;
	\end{enumerate}
\end{tbox}