\subsection{\textbf{Теорема об изменении импульса с.м.т.} Условия сохранения импульса.}

Запишем теорему об изменении импульса для одной материальной точки:
\begin{itemize}
	\item \(\vec{p} = m\vec{v}\) -- определение импульса;
	\item \(\dd{\vec{p}}{t} = \vec{F}\) -- II закон Ньютона (Теорема об изменении импульса материальной точки).
\end{itemize}

\textbf{Импульсом системы материальных точек} называется сумма импульсов всех точек, входящих в систему.
\[\vec{p} = \sum_{i=1}^{n} \vec{p}_i\]

Импульс системы является характеристикой поступательного движения системы как целого.
\begin{align*}
	\dd{\vec{p}_1}{t} &= \vec{f}_{21} + \vec{f}_{31} + \dots + \vec{f}_{i1} + \dots + \vec{f}_{n1} + \vec{F}_1; \\
	\dd{\vec{p}_i}{t} &= \vec{f}_{1i} + \dots + \vec{f}_{(i-1) \, 1} +\vec{f}_{(i+1) \, 1} + \dots + \vec{f}_{ni} + \vec{F}_i; \\
	\dd{\vec{p}_n}{t} &= \vec{f}_{1n} + \dots + \vec{f}_{in} + \dots + \vec{f}_{(n-1)n} + \vec{F}_n;
\end{align*}

Сложим попарно внутренние силы и сократим: \(\vec{f}_{ik} + \vec{f}_{ki} = 0\).
\[\sum_{i=1}^{n} \dd{\vec{p}_i}{t} = \sum_{i=1}^{n} \vec{F}_i\]

Где $\vec{F}_{\text{внеш}} = \sum_{i=1}^{n} \vec{F}_i$ -- суммарная внешняя сила. Из чего следует:
\[\sum_{i=1}^{n} \dd{\vec{p}_i}{t} = \dd{}{t} \sum_{i=1}^{n} \vec{p}_i = \dd{\vec{p}}{t} \quad\Rightarrow\quad \boxed{\dd{\vec{p}}{t} = \vec{F}_\text{внеш}}\]

Запишем в интегральной форме:
\[\boxed{\vec{p}(t) - \vec{p}(t_0) = \int_{t_0}^{t} \vec{F}_{\text{внеш}} \, dt}\]

\begin{tbox}{Замечания}
	\begin{enumerate}
		\item Теорема справедлива и в НСО\footnote{Неинерциальная система отсчета (НСО) — система отсчета, движущаяся с ускорением относительно инерциальной. Простейшими НСО являются системы, движущиеся ускоренно прямолинейно, и вращающиеся системы.} (нужно учесть силы инерции);
		\item Описывает поступательное движение в целом и не включает внутреннюю силу;
	\end{enumerate}
\end{tbox}

\subsubsection*{Условия сохранения импульса}

Импульс сохраняется, если сумма внешних сил равна нулю:
\[
\vec{p} = \text{const}, \quad \text{если} \quad \vec{F}_{\text{внеш}} = 0.
\]

\begin{enumerate}
	\item \textbf{Замкнутая (изолированная) система:}
	\[
	\vec{F}_i^{\text{внеш}} = 0
	\]
	\item \textbf{Сбалансированная система:}
	\[
	\vec{F}_i^{\text{внеш}} \neq 0, \quad \text{но} \quad \sum_{i=1}^n \vec{F}_i^{\text{внеш}} = 0;
	\]
	\item \textbf{Сохранение проекции импульса:} Если внешняя сила вдоль оси $x$ равна нулю, то соответствующая проекция импульса сохраняется:
	\[
	F_x^{\text{внеш}} = 0 \quad \Rightarrow \quad p_x = \text{const}.
	\]
\end{enumerate}

\textbf{Случай мгновенного взаимодействия ($\Delta t \to 0$):}

Если внешняя сила конечна, а время взаимодействия стремится к нулю ($\Delta t = t - t_0 \to 0$), то изменение импульса также стремится к нулю:
\[
\Delta \vec{p} = \int_{t_0}^{t} \vec{F}^{\text{внеш}} \, dt \to 0.
\]

Это означает, что при очень коротких взаимодействиях (например, ударах) импульс системы можно считать приблизительно постоянным, даже если внешние силы не равны нулю.

\textbf{Примеры:}
\begin{itemize}
		\item Движение ракеты в открытом космосе (замкнутая система);
		\item Столкновение бильярдных шаров на столе (силы реакции опоры компенсируются);
		\item Движение снаряда в горизонтальной плоскости (если пренебречь сопротивлением воздуха).
\end{itemize}